\documentclass[a4paper,12pt,onside]{report}
\usepackage[utf8]{inputenc}
\usepackage{a4wide,url}   
\usepackage{amsmath}  
\usepackage{amsfonts}
\usepackage{amssymb}
\usepackage{amsthm} 
\usepackage[french]{babel}
\usepackage{bbold}
\usepackage{bbm}
\usepackage{calligra}
\usepackage{color}
\usepackage{dsfont}
\usepackage{fancyhdr}
\usepackage[Glenn]{ fncychap}
\usepackage[T1]{fontenc}
\usepackage[left=3.5cm,right=3.5cm,top=3.5cm,bottom=3.5cm]{geometry}
\usepackage{graphics,graphicx,subfigure,wrapfig,picinpar}
\usepackage{lipsum}
\usepackage{lmodern}
\usepackage{microtype}
\usepackage{xspace}
\usepackage{xcolor}
\usepackage{wasysym}
\usepackage{fancyhdr}
\usepackage{epstopdf}
\usepackage{datetime}
\usepackage{hyperref}
\usepackage[all,cmtip]{xy}
\usepackage{listings}
\usepackage{caption}
\usepackage{pdfpages}
\usepackage{mathpazo} % use palatino
\usepackage[scaled]{helvet} % helvetica

\newtheorem {theorem}{Théorème}[chapter]
\newtheorem {lemme}{Lemme}[chapter]
\newtheorem {corollary}{Corollaire}[chapter]
\newtheorem {proposition}{Proposition}[chapter]
\newtheorem {definition}{Définition}[chapter]
\newtheorem*{ proof }{Preuve}
\newtheorem{remark}{Remarque}[chapter]


\newenvironment{prof}[1][Preuve]{\textbf{#1.} }{\ \rule{0.5em}{0.5em}}
%\theoremstyle{plain}
\newtheorem{exercise}{Exercice}[chapter]
\newtheorem{coro}{Corollaire}[chapter]
\newtheorem{ex}{Exemple}[chapter]
\newtheorem{theodef}{Th\'eor\`eme et D\'efinition}[chapter]
\newtheorem{rap}{Rappel}[chapter]
\newtheorem{thm}{Théoréme}[chapter]
\newtheorem{prop}{Proposition}[chapter]
\newtheorem{pro}{Propriétés}[chapter]
\newtheorem{lem}{lemme}[chapter]
\newtheorem{dfn}{Définition}

\newtheorem{Post}{Postulat}



\DeclareGraphicsRule{.tif}{png}{.png}{`convert #1 `dirname #1`/`basename #1 .tif`.png}



\usepackage{eucal} 
\addtolength{\hoffset}{-1.0cm} 
\addtolength{\textwidth}{2.0cm}%1.6avant
\addtolength{\voffset}{-1.0cm} 
\addtolength{\textheight}{2.0cm}


\makeatletter
\newcommand{\thechapterwords}
{ \ifcase \thechapter\or Premier\or Deux\or Trois\or Quatre\or
Cinq\or
Six\or Sept\or Huit\or Neuf\or Dix\or Onze\fi}
\def\thickhrulefill{\leavevmode \leaders \hrule height 1ex \hfill \kern \z@}
\def\@makechapterhead#1{
\vspace*{15\p@}
{\parindent \z@ \centering \reset@font
\thickhrulefill\quad
\scshape \@chapapp{} \thechapterwords
\quad \thickhrulefill
\par\nobreak
\vspace*{15\p@}
\interlinepenalty\@M
\hrule
\vspace*{15\p@}
\Huge \bfseries #1\par\nobreak
\par
\vspace*{15\p@}
\hrule
\vskip 60\p@
}}
\def\@makeschapterhead#1{
\vspace*{15\p@}
{\parindent \z@ \centering \reset@font
\thickhrulefill
\par\nobreak
\vspace*{15\p@}
\interlinepenalty\@M
\hrule
\vspace*{15\p@}
\Huge \bfseries #1\par\nobreak
\par
\vspace*{15\p@}
\hrule
\vskip 60\p@
}}
\def\@makechapterhead#1{
\vspace*{15\p@}
{\parindent \z@ \centering \reset@font
\thickhrulefill\quad
\scshape \@chapapp{} \thechapterwords
\quad \thickhrulefill
\par\nobreak
\vspace*{15\p@}
\interlinepenalty\@M
\hrule
\vspace*{15\p@}
\Huge \bfseries #1\par\nobreak
\par
\vspace*{15\p@}
\hrule
\vskip 60\p@
}}
\frenchspacing \pagestyle{headings}
\usepackage{fancyhdr}
\pagestyle{fancy}

\usepackage [dvips]{epsfig}

\renewcommand{\sectionmark}[1]{\markboth{#1}{}}
\renewcommand{\sectionmark}[1]{\markright{\thechapter\ #1}}


\renewcommand{\sectionmark}[1]{\markboth{#1}{}}

\fancyhf{}

\fancyhead[L,R]{\bfseries\thepage}

\fancyhead[L]{\bfseries\rightmark} % Left Odd
\fancyhead[R]{\bfseries\leftmark} % Right Even
\renewcommand{\headrulewidth}{1pt}

\addtolength{\headheight}{1pt} 

\renewcommand{\footrulewidth}{0.5pt}

\fancypagestyle{plain}{ % pages de tetes de chapitre
\fancyhead{} % supprime l'entete
%\fancyfoot{} %supprime le pied de page
\renewcommand{\headrulewidth}{0pt}
}
\newcommand{\clearemptydoublepage}{%
\newpage{\pagestyle{plain}\cleardoublepage}}

\rhead{\textbf{\thepage}}
\lhead{\textsl{\leftmark}}
\fancyfoot[LO]{\tiny \emph{\textbf{Schémas monotones discrets pour l'équation de Schrödinger}}} 

\usepackage{newlfont}

%\usepackage[active]{srcltx}
\hfuzz2pt
\newlength{\defbaselineskip}
\setlength{\defbaselineskip}{\baselineskip}
\newcommand{\setlinespacing}[1]%
{\setlength{\baselineskip}{#1 \defbaselineskip}}
\newcommand{\doublespacing}{\setlength{\baselineskip}%
{1.5 \defbaselineskip}}
\newcommand{\singlespacing}{\setlength{\baselineskip}{\defbaselineskip}}


\newcommand{\Z}{\mathbb Z}
\newcommand{\R}{\mathbb R}
\newcommand{\C}{\mathbb C}
\newcommand{\N}{\mathbb N}
\newcommand{\F}{\mathbb F}
%\newcommand{\1}{1 \! \! {\rm I}}
\newcommand{\cc}{\mathcal{C}}
\newcommand{\A}{\mathcal{A}}
\newcommand{\X}{\mathcal{X}}
\newcommand{\I}{\mathcal{I}}
\newcommand{\el}{\mathcal{L}}
\newcommand{\G}{\mathcal{G}}
\newcommand{\B}{\mathcal{B}}
\newcommand{\f}{\mathcal{F}}
\newcommand{\E}{\mathcal{E}}
%\newcommand{\H}{\mathcal{H}}
\newcommand{\M}{\mathcal{M}}
\newcommand{\W}{\mathcal{W}}
\newcommand{\n}{\mathcal{N}}
\newcommand{\ho}{\hbox{\rm Hom}}
\newcommand{\Q}{l \! \! \! Q}
\newcommand{\0}{/ \! \! \! 0}
\newcommand{\g}{\mathfrak g}
\newcommand{\h}{\mathfrak h}

\usepackage{amssymb,textcomp}

\def\card{{\rm card }}

\usepackage{stmaryrd}
\begin{document}
\begin{titlepage}
\begin{center}

\begin{center}
\textsc{Memoire de Master}\\
\vspace{0.5cm}
\textsc{Math\'ematiques Fondamentales et Applications}\\
\vspace{1.2cm}
\textsc{\underline{Thème}}\\
\vspace{0.8cm}
\end{center}

\rule{\linewidth}{1mm}\\
\begin{center}
\textsc{\textbf{ \color{blue}{Schémas monotones discrets pour l'équation de Schrödinger}}}
\end{center}
\rule{\linewidth}{1mm}
\vspace{1cm}
\begin{flushright}
\textbf{Présenté par:}\\
\vspace{0.2cm}
\textcolor{magenta}{Kenneth ASSOGBA}\\
\textcolor{black}{kennethassogba@gmail.com}\\
\vspace{1cm}
\textbf{\textcolor{black}Sous la direction de :}\\
\vspace{0.2cm}
\textcolor{magenta}{\textbf{Prof.} Julien SALOMON}\\
\textsc{Sorbonne Université, UPMC Univ Paris 06,\\
	UMR 7598, Laboratoire Jacques-Louis Lions,\\
	Paris, France}\\
\vspace{0.2cm}
\end{flushright}
\vspace{2cm}
\vspace{4cm}
\begin{center}
\noindent \textbf{Année Acad\'emique : 2018-2019}
\end{center}
%\vspace{1cm}
%\rule{\linewidth}{1mm}
\end{center}
\end{titlepage}
%\newpage
\pagenumbering{roman}
%\chapter*{D\'edicaces}\addcontentsline{toc}{chapter}{Dédicaces}
\vfill
\begin{center}
\`A mon frère feu \textbf{Daniel DOTSE},\\
\vspace{1.5cm}
\textbf{\`A mes parents.}
\end{center}
\vfill

%\chapter*{Remerciements}\addcontentsline{toc}{chapter}{Remerciements}
\begin{flushleft}
Je remercie tout d'abord DIEU qui m'a donné la force et le courage d'accomplir ce modeste travail.
\end{flushleft}
\[\]
Je tiens à exprimer toute ma reconnaissance à mon Directeur de mémoire le Professeur Julien SALOMON qui a bien voulu m'accorder sa confiance en me permettant de travailler sur ce thème. Sa patience et son soutien m'ont été précieux pour la conduite de ce travail dans de très bonnes conditions.\\\\
Je remercie l'Institut de Mathématiques et de Sciences Physiques (IMSP), l'établissement qui m'a servi de cadre adéquat pendant ces deux ans d'études de Master. Je remercie le Directeur, Professeur Léonard TODJIHOUNDE, le Directeur Adjoint, Professeur Carlos OGOUYANDJOU, les enseignants missionnaires comme locaux ainsi que tout le personnel administratif de l'Institut.\\\\
Je tiens à dire merci pour le projet Centre d'Excellence Africain en Sciences Mathématiques et Applications (CEA-SMA) pour son soutient financier. Je remercie également le coordonnateur de ce projet Professeur Joël TOSSA et son coordonnateur adjoint Professeur Aboubakar MARCOS.\\\\
Mes remerciements vont également à tous mes amis et camarades ainsi qu'à tous ceux qui ont contribué de prêt ou de loin à la réalisation de ce mémoire.\\\\\\
Enfin j'exprime ma profonde gratitude à ma très chère famille, principalement mes parents.
\chapter*{Résumé}\addcontentsline{toc}{chapter}{Résumé}
\Large
\begin{flushleft}
\textbf{\underline{Résumé}}
\end{flushleft}
\normalsize
Les problèmes de contrôle optimal sur les systèmes quantiques suscitent un vif intérêt, aussi bien pour les questions fondamentales que pour les applications existantes et futures. Un problème important est le développement de méthodes de construction de contrôles pour les systèmes quantiques. Une des méthodes couramment utilisée est la méthode de Krotov initialement proposée dans un cadre plus général dans les articles de V.F. Krotov et I.N. Feldman (1978 \cite{Krotov1}, 1983 \cite{Krotov2}). Cette méthode a été utilisée pour développer une nouvelle approche permettant de determiner des contrôles optimaux pour les systèmes quantiques dans \cite{Tannor} et dans de nombreux autres travaux de recherche: \cite{Zhu}, \cite{Maday} et \cite{Salomon} notamment. Leur mise en œuvre numérique repose souvent sur des discrétisations liées à des développement limités. Cette approche entraîne cependant parfois des instabilités numériques. Nous présentons ici plusieurs méthodes de discrétisation temporelle qui permettent de résoudre ce problème en conservant au niveau discret la monotonie des schémas.\\\\
\textbf{Mots-clés}:  contrôle optimal, contrôle quantique, schémas monotones, methode de Krotov.
\chapter*{Abstract}\addcontentsline{toc}{chapter}{Abstract}
\Large\textbf{\underline{Abstract}}\\\\\normalsize
Mathematical problems of optimal control in quantum systems attract high interest in connection with fundamental questions and existing and prospective applications. An important problem is the development of methods for constructing controls for quantum systems. One of the commonly used methods is the Krotov method initially proposed beyond quantum control in the articles by V.F. Krotov and I.N. Feldman (1978 \cite{Krotov1}, 1983 \cite{Krotov2}). The method was used to develop a novel approach for finding optimal controls for quantum systems in \cite{Tannor}, and in many works of various scientists: \cite{Zhu}, \cite{Maday} and \cite{Salomon} especially. However, the properties of the discrete version of these procedures have not been yet tackled with.
We present here a stable time and space discretization which preserves the monotonic properties of the monotonic algorithms.\\\\
\textbf{Keywords}: optimal control, quantum control, monotonically convergent algorithms, Krotov method

\chapter*{Notations}\addcontentsline{toc}{chapter}{Notations}
$
\begin{array}{ll}
\Omega & \mbox{Espace des configurations}\\\\
\mathcal{H} & \mbox{Espace de Hilbert correspondant a un systeme quantique.}\\
& \mbox{On prends generalement } \mathcal{H} = \C^n \mbox{ ou } L^2(\Omega, C^n) \mbox{ (a verifier)}\\\\
\mathcal{H^*} & \mbox{Dual de } \mathcal{H}\\\\
\lVert\cdot\rVert &  \mbox{norme associée à } \mathcal{H}\\\\
\langle\cdot,\cdot\rangle & \mbox{produit hermitien associé à } \mathcal{H}\\
& \langle\psi,\phi\rangle=\sum_{k=1}^{n} \psi^*_k \phi_k \mbox{ en dimension finie}\\
& \langle\psi,\phi\rangle=\int_{\Omega} \psi^*(x) \phi(x) dx \mbox{ en dimension infinie}\\
&\mbox{on note indifféremment } \langle\psi,\phi\rangle = \langle\psi||\phi\rangle = \langle\psi|\phi\rangle\\\\
| \phi \rangle & \mbox{notation ket de Dirac}\\
& \mbox{dans ce memoire on note indifféremment } \phi = | \phi \rangle \in \mathcal{H}\\\\
\langle \psi | & \mbox{notation bra de Dirac}\\
& \mbox{dans ce memoire on note indifféremment } \psi = \langle  \psi | \in \mathcal{H^*}\\\\
L^2(\Omega) & \mbox{ L'espace des fonctions de carré intégrable sur } \Omega\\\\
H^n(\Omega) & \mbox{ L’espace des fonctions de }L^2(\Omega) \mbox{ dont les dérivées partielles jusqu’à }\\
&\mbox{ l’ordre } n \mbox{ appartiennent à } L^2(\Omega)\\\\
|\cdot| & \mbox{ Valeur absolue}\\\\
H_u(z) &  \mbox{ Matrice hessienne de la fonction } u \mbox{ au point } z\\\\
\left(\cdot,\cdot \right) & \mbox{ Produit scalaire de }\mathbb{R}^d\\\\
\nabla u & \mbox{ Gradient de la fonction } u\\\\
A^T & \mbox{ Transposée de la matrice } A\\\\
\Re & \mbox{partie réelle}\\\\
\Im & \mbox{partie imaginaire}
\end{array}
$

\tableofcontents%\addcontentsline{toc}{chapter}{Table des matières}
\listoffigures\addcontentsline{toc}{chapter}{Table des figures}
%\listoftables\addcontentsline{toc}{chapter}{Liste des tableaux}
\newpage
\newpage
\pagenumbering{arabic}
\chapter*{Introduction}\addcontentsline{toc}{chapter}{Introduction}

\section*{Origines de la mécanique quantique}
A la fin du XIXe siècle, les diverses branches de la physique s'intégraient dans un édifice cohérent, basé sur l'étude de deux types d’objets distincts, la matière et le rayonnement:
\begin{itemize}
	\item La matière est faite de corpuscules parfaitement localisables dont le mouvement peut être décrit par la mécanique de Newton. Les grandeurs physiques associées à ces corpuscules s’expriment en fonction des composantes de la position et de l’impulsion qui sont les variables dynamiques fondamentales.
	\item Le rayonnement est gouverné par les lois de l'électromagnétisme de Maxwell. Ses variables dynamiques sont les composantes en chaque point de l’espace des champs électrique et magnétique.
\end{itemize}
Le succès de la physique était à cette époque impressionnant et tous les phénomènes connus trouvaient leur explication dans le cadre de ce programme classique.\\\\

A l’aube du XXe siècle et avec l’essor des progrès technologiques, les physiciens se trouvèrent tout à coup confrontés à des phénomènes nouveaux pour lesquels les prévisions de la théorie classique sont en désaccord flagrant avec l'expérience. Il fallait donc jeter les bases d’une nouvelle théorie susceptible de pallier les insuffisances de la conception classique.

\section*{Contrôle optimal et optimisation numérique}
L'objet de notre étude est un système quantique, modélisé entre deux mesures par l'équation de Schrödinger:
\begin{equation} \label{schrodinger}
i \hbar \dfrac{\partial }{\partial t} \psi (x,t) = H\psi (x,t)
\end{equation}
En vue de modéliser les interactions onde-matière à l'échelle atomique, nous introduisons un contrôle, généré par un dipôle électrostatique de moment dipolaire $\mu (x)$, émettant un champs (électrique) laser, d'amplitude $\varepsilon (t)$ dépendant du temps.\\
La dynamique du système est désormais donnée par:
\begin{equation} \label{eq1}
\begin{cases}
i \hbar \dfrac{\partial }{\partial t} \psi (x,t) &= H\psi (x,t)-\mu(x)\varepsilon(t)\psi (x,t) \\
\psi (x,t=0) &= \psi_0(x)
\end{cases}
\end{equation}
$H$ étant un opérateur hermitien, défini par:
$$
H = H_0 + V = -\dfrac{1}{2m} \varDelta + V
$$
En posant:
\begin{equation}
A(\psi(t),\varepsilon(t))= -i(H-\mu(x)\varepsilon(t))\psi (x,t)
\end{equation}
On se ramène au problème de Cauchy
\begin{equation} \label{chauchy1}
\begin{cases}
\dot{\psi}(t) &= A(\psi(t),\varepsilon(t))\\
\psi (t=0) &= \psi_0
\end{cases}
\end{equation}
Nous nous posons maintenant deux questions.
\subsection*{Problème de contrôlabilité}
Un système est dit contrôlable si on peut le ramener à tout état prédéfini au moyen d’un contrôle. Plus précisément on pose la définition suivante.
\begin{dfn}
On dit que le système \eqref{chauchy1} est contrôlable (ou commandable) si pour tous les états $\psi_0 \in \mathcal{H}$ , $\psi_{cible} \in \mathcal{H}$ , il existe un temps fini $T$ et un contrôle admissible $\varepsilon(.) : [0, T] \longrightarrow \R$ tel que $\psi_{cible} = \psi(T, \psi_0, \varepsilon(.))$.
\end{dfn}
\begin{figure}[H]
	\caption{Problème de contrôlabilité}
	\centering
	\includegraphics[scale=0.6]{images/controle.png}
\end{figure}
Si la condition précédente est remplie, existe-t-il un contrôle joignant $\psi_0$ à $\psi_{cible}$ , et qui de plus minimise une certaine fonctionnelle $J(\varepsilon)$ ?
\subsection*{Contrôle optimal}
La fonctionnelle $J(\varepsilon)$ est un critère d’optimisation, on l’appelle le coût . Par exemple ce coût
peut être égal au temps de parcours; dans ce cas c’est le problème du temps minimal.
%Nous voulons construire un contrôle d'amplitude "raisonnable" afin d’amener le système d'un état initial $\psi_0$ à un état cible $O\psi(T)$. 
%$O$ étant l'observable décrivant l'état cible. \\\\
\begin{figure}[H]
	\caption{Problème de contrôle optimal}
	\centering
	\includegraphics[scale=0.6]{images/controleoptimal.png}
\end{figure}

On considère ainsi une fonctionnelle $J$
\begin{equation}
J(\varepsilon) = \langle \psi(T)|O|\psi(T) \rangle - \alpha \int_{0}^{T}\varepsilon^2(t)dt \quad \alpha \in \R_+
\end{equation}
et on se pose le problème: Trouver $\varepsilon$ tel que $\varepsilon$ résout
$$ \max_{\varepsilon \in L^2(0,T)} J(\varepsilon)$$
Au maximum de la fonctionnelle $J(\varepsilon)$, les équations de Euler-Lagrange sont satisfaites. Le Lagrangien du système est donné par :
\begin{equation} \label{lagrangien}
L(\psi,\varepsilon,\chi)= J(\varepsilon)\\
-2\Re \bigg\{ \int_{0}^{T}\langle \chi (t)|\partial_{t}+i(H_0+V-\mu(x)\varepsilon(t))|\psi(t) \rangle dt \bigg\}
\end{equation}
\section*{Schémas monotones}
Une stratégie éfficace de résolution de ces équations est donnée par une classe d’algorithmes relevant du contrôle quantique, les schémas monotones. Ils ont étés introduits en 1992 par David Tannor, Vladimir Kazakov et V. Orlov,  \cite{Tannor}, sur la base des travaux de Krotov \cite{Krotov1}, \cite{Krotov2}. Une amélioration a ensuite été proposée par Wusheng Zhu et Herschel Rabitz \cite{Zhu} en 1998. Une généralisation est donnée par Yvon Maday et Gabriel Turinici en 2003 \cite{Maday}.\\\\

Comment construire une discrétisation temporelle puis spaciale de ces algorithmes qui préserve la propriété de monotonie?\\\\

Dans le Chapitre Premier, nous introduisons la mécanique quantique en trois postulats et présentons le cadre général du contrôle quantique. Le Chapitre Deux est dédié aux schémas monotones pour l'équation de Schrödinger.
Différentes discrétisations de ces schémas sont proposées dans le Chapitre Trois.
\newpage
\chapter{Mécanique quantique et contrôle quantique}

\section{La mécanique quantique en trois postulats}
\subsection{Premier postulat de la mécanique quantique}
\subsubsection{Fonction d'onde}
En mécanique quantique, l'état d'un système donné, est donné par son vecteur d'état $|\psi (t)\rangle$. 
%On utilise la notation bra-ket introduite par Dirac \footnote{Paul Dirac, Les Principes de la mécanique quantique [« The Principles of Quantum Mechanics »] (1re éd. 1930) } pour représenter les états quantiques de manière concise et simple. Le vecteur $\psi$ est ainsi noté $|\psi \rangle$ et appelé \textbf{ket}, tandis que son vecteur dual est appelé \textbf{bra} et noté $\langle \psi |$. \\
L'espace des états dépend du système considéré. Par exemple, dans le cas le plus simple où le système n'a pas de spin, les états quantiques sont des fonctions $\psi$ :
$$
\begin{aligned}
	\psi :\quad \quad \mathbb {R} ^{3}&\rightarrow \mathbb{C} \\(x,y,z)&\mapsto \psi (x,y,z)
\end{aligned}
$$
telles que l'intégrale  $\int _{\mathbb{R} ^{3}}|\psi (\mathbf{r} )|^{2}d\mathbf{r}$ converge. Dans ce cas, $\psi$ est appelée la fonction d'onde du système.
\subsubsection{Cas d’une particule dans l’espace a une dimension}
La probabilité pour que la particule soit dans l’intervalle $[a,b]$ est donnée par l’aire de la courbe située entre $x=a$ et $x=b$ (figure si possible)
\begin{equation}
\int_a^b dP(x)= \int_a^b |\psi(x,t)|^2dx
\end{equation}
Il est impossible de connaître avec précision la position de la particule a un instant t. On ne peut que connaître la probabilité $dP(x)$ pour qu’elle soit entre $x$ et $x+dx$, soit:
\begin{equation}
dP(x)=|\psi(x,t)|^2dx=\psi(x,t)\overline{\psi(x,t)}dx
\end{equation}
La particule doit être quelque part sur l’axe $X’OX$, par conséquent:
\begin{equation}
\int_{-\infty}^{+\infty}|\psi(x,t)|^2dx=1
\end{equation}
pour tout $t$. $\psi$ est donc de carre sommable.
La densite de probabilite est donnee par
\begin{equation}
\dfrac{dP(x,t)}{dx}=|\psi(x,t)|^2=\rho(x,t)
\end{equation}

\subsubsection{Cas d’une particule dans l’espace à trois dimensions}
On a 
\begin{equation}
\int dP(\vec{r},t)=\iiint_{espace}|\psi(x,t)|^2d^3r=1
\end{equation}

Ou $d^3r$ représente l'élément de volume donnée par: $$d^3r=dxdydz=r^2sin\theta dr d\theta d\varphi$$

\subsubsection{Cas de N particules}
L’espace le mieux adapté à la description des systèmes en physique quantique est un espace $\Omega$, nommé espace des configurations qui représente l’ensemble de toutes les configurations possibles du système. Par exemple, dans le cas d’un système à $N$ particules isolées et sans contraintes, l’espace des configurations est $\Omega = \R^{3N}$ et $\psi(x,t) \in L^2(\Omega, \C)$.

\begin{Post}
	A tout système quantique correspond un espace de Hilbert complexe $\mathcal{H}$, tel que l’ensemble des états accessibles au système soit en bijection avec la sphère unité de $\mathcal{H}$.
\end{Post}

\subsection{Observables et deuxième postulat}
Une observable est l'équivalent en mécanique quantique d'une grandeur physique en mécanique classique, comme la position, la quantité de mouvement, l'énergie, etc. Une observable est formalisée mathématiquement par un opérateur hermitien (endomorphisme autoadjoint) sur $\mathcal{H}$ (chaque état quantique est représenté par un vecteur de $\mathcal{H}$). 
%Cet operateur permet de décomposer un état quantique quelconque $|\psi \rangle$ en une combinaison linéaire d'états propres,
\begin{ex}
	Exemples d'observables
	\begin{enumerate}
		\item la position: $X$
		\item l'énergie potentielle: $V$
		\item la quantité de mouvement: $P=-i\hbar \nabla$
		\item l'énergie cinétique:
		$H_0=\dfrac{P \cdot P}{2m}=-\dfrac{\hbar ^{2}}{2m}\nabla^{2}$
		\item l'énergie totale, appelé hamiltonien:
		$H=H_0+V$
	\end{enumerate}
\end{ex}
%En mécanique quantique, une grandeur ne prend une valeur déterminée que lors d’une mesure:
\begin{Post}
	A toute grandeur physique représentée par l'observable $\mathcal{A}$ correspond un opérateur $A$ auto-adjoint sur $\mathcal{H}$, vérifiant la propriété suivante : le résultat de la mesure d’une grandeur physique $\mathcal{A}$ ne peut être qu’un élément du spectre de $A$.
\end{Post}
Les valeurs propres sont les valeurs pouvant résulter d'une mesure idéale de cette propriété, les vecteurs propres étant l'état quantique du système immédiatement après la mesure et résultant de cette mesure. Ce postulat peut donc s'écrire :
$$
A|\alpha _{n}\rangle =a_{n}|\alpha _{n}\rangle 
$$
où $A$, $|\alpha_{n}\rangle$ et $a_n$ désignent, respectivement, l'observable, le vecteur propre et la valeur propre correspondante.\\
Les états propres de tout observable $A$ forment une base orthonormée dans l'espace de Hilbert $\mathcal{H}$.
Cela signifie que tout vecteur $|\psi (t)\rangle$ peut se décomposer de manière unique sur la base de ces vecteurs propres $|\phi_{i}\rangle$:
$$
|\psi \rangle =c_{1}|\phi _{1}\rangle +c_{2}|\phi _{2}\rangle +...+c_{n}|\phi _{n}\rangle +...
$$
La moyenne des mesures de $\mathcal{A}$ est quant à elle égale à $\langle\psi|A|\psi\rangle$ où la notation $\langle\cdot|A|\cdot\rangle$ est définie par :
\begin{equation}
\langle\psi|A|\chi\rangle = \int_{\Omega} \bar{\psi}A\chi
\end{equation}
%ou $\psi$ et $\chi$ sont des fonctions de $L^2(\Omega, \C)$ et $A$ un opérateur arbitraire défini de $L^2(\Omega, \C)$ dans lui-même.
\begin{ex}
	La position moyenne sur l’axe des abcisses:
	\begin{equation*}
	\langle X \rangle = \int_{- \infty}^{- \infty} x |\psi(x,t)|^2 dx
	\end{equation*}
\end{ex}

\subsection{Equation de Schrödinger et troisième postulat}
L'équation de Schrödinger, proposée par le physicien autrichien Erwin Schrödinger en 1952, est une équation fondamentale en mécanique quantique. Elle décrit l'évolution dans le temps d’une particule massive non relativiste, et remplit ainsi le même rôle que la relation fondamentale de la dynamique en mécanique classique $\dfrac{dp}{dt}=F$.
\begin{Post}
	Entre deux mesures, l’évolution de l’état est régie par l’équation de Schrödinger
	\begin{equation} \label{sch1}
	i\hbar \dfrac{\partial }{\partial t} \psi (x,t) = H\psi (x,t)
	\end{equation}
\end{Post}
%\subsubsection{Exemple de resolution de \eqref{sch1}: cas d'une particule libre}
Considerons le cas d'une particule libre ($V=0$) et menons notre étude sur $\R^m$ $(m \geq 1)$ nous étudions alors l'équation
\begin{equation} \label{chauchy2}
u'(t)=Au(t)
\end{equation}
où $A$ est un operateur sur $E=L^2(R^m)$ defini par 
$$
Au=ik\Delta u=ik(\partial_{1}^2u+...+\partial_{m}^2u)
$$
$k$ une constante reelle. Le domaine de $A$ est l'ensemble des $u \in L^2(R^m)$ tel que $\Delta u$ (au sens des distributions) appartient a $L^2$.
\begin{definition}
	On dit que le probleme de Cauchy pour l'equation \eqref{chauchy2} est bien pose si les deux hypotheses suivantes sont satisfaites:
	\begin{enumerate}
		\item[(a)] \textbf{Existance et unicite de solutions}: Il existe un sous espace dense $D$ de $E$ tel que, pour tout $u_0 \in D$, il existe une unique solution $u(.)$ de \eqref{chauchy2} avec $u(0)=u_0$
		\item[(b)] \textbf{La solution dépend de façon continue des données} Il existe une fonction non décroissante, non négative $C(t)$ tel que
		$$
		||u(t)||\leq C(t) ||u(0)||
		$$
	\end{enumerate}
\end{definition}
En appliquant la transformee de Fourier a \eqref{chauchy2}, nous obtenons
\begin{align*} 
\mathcal{F}[Au](\xi) &= \mathcal{F}[ik\Delta u](\xi) \\ 
&= ik\mathcal{F}[\Delta u](\xi) \\
&= -ik|\xi|^2 \mathcal{F}[u](\xi) \\
\mathcal{F}[Au](\xi) &= -ik|\xi|^2 \tilde{u}(\xi)
\end{align*}
Alors si $u_0 \in D(A)$ nous obtenons une solution:
\begin{equation}
u(t)=\mathcal{F}^{-1}(exp(-ik|\xi|^2 t)\mathcal{F}[u_0](\xi))
\end{equation}
Puisque $D(A)$ est dense dans $E$, on deduis que \eqref{chauchy2} admet une unique solution sur $E$.\\
Ensuite, observons que si $u \in D(A)$, alors
\begin{align*}
(Au,u) &= (\mathcal{F}[Au],\mathcal{F}[u])\\
&= ik\int|\xi|^2 |\tilde{u}|^2
\end{align*}
Ainsi, $\Re (Au,u)=0$\\
Par consequent, si $u(.)$ est une solution de \eqref{chauchy2}, nous avons:
\begin{align*}
\partial_{t}||u(t)||^2 &= 2 \Re (u'(t),u(t))\\
&= 2 (Au(t),u(t))\\
&=0
\end{align*}
Ainsi, $||u(t)||$ est constante:
\begin{equation} \label{normecons}
||u(t)||=||u(0)|| \quad \forall t \in \R
\end{equation}
En conclusion, le probleme de Cauchy pour \eqref{chauchy2} est bien pose pour tout $t$.\\
En outre, \cite{Fattorini} generalise ce resultat sur $E=L^p (\R^m)$ avec $1 \leq p < m$.

\section{Controle quantique}
On rappelle que l'état $\psi(t)$ d'un systeme quantique evolue conformement a l'équation de Schrödinger:
\begin{equation}
i \hbar \dfrac{\partial }{\partial t} \psi (x,t) = H\psi (x,t),\quad \quad \psi(t=0)=\psi_0
\end{equation}
ou $H$ est un endomorphisme auto-adjoint sur $\mathcal{H}$ appelé Hamiltonien interne du systeme. $\hbar$ est la constante de Planck. Dans la suite, nous travaillons en unités atomiques, c'est-a-dire $\hbar=1$ et l'etat initial a une norme unitaire, $||\psi_0||^2 \equiv \langle \psi_0|\psi_0 \rangle =1$.
\subsection{Exemple de modele bilineaire (BLM)}
Pour simplifier, nous considerons un systeme quantique de dimension finie. C'est une approximation appropriee dans de nombreuses situations pratiques. Pour un systeme quantique de dimension $N$, $\mathcal{H}$ est un espace de Hilbert complexe de dimension $N$. Les valeurs propres $(\phi_{i})_{1 \leq i \leq N}$ de $H$ forment une base orthogonale de $\mathcal{H}$.\\
Dans de nombreuses situations, le controle du systeme peut-etre realise par un ensemble de fonctions de controle $\varepsilon_{k}(t) \in \R$ couplees au systeme via une famille d'operateurs hermitions independants du temps $\{H_k, k=1,2,...\}$. L'Hamiltonien total $H+\sum_{k} \varepsilon_{k}(t)H_k$ determine alors la nouvelle dynamique du systeme.
\begin{equation} \label{schcon}
i \dfrac{\partial }{\partial t} \psi (x,t) = (H+\sum_{k} \varepsilon_{k}(t)H_k)\psi (x,t)
\end{equation}
Le but typique d'un probleme de conrole quantique defini sur le systeme \eqref{schcon} est de trouver en temps fini $T > 0$ un ensemble de controles admissibles $u_k(t) \in \R$ qui conduit le systeme d'un etat initial $\psi_0$ a un etat prescrit $\psi_{cible}$.
\subsection{Cas d'un champs laser}
En vue de modéliser les intéractions onde-matière a l'échelle atomique, nous introduisons un contrôle, généré par un dipole électrostatique de moment dipolaire $\mu (x)$, émetant un champs (électrique) laser, d'amplitude $\varepsilon (t)$ dépendant du temps.\\
La dynamique du systeme est désormais donnée par:
\begin{equation}
\begin{cases}
i \dfrac{\partial }{\partial t} \psi (x,t) &= H\psi (x,t)-\mu(x)\varepsilon(t)\psi (x,t) \\
\psi (x,t=0) &= \psi_0(x)
\end{cases}
\end{equation}

\section{Discrétisation temporelle}
Choisissons deux paramètres de discrétisation temporelle $N$ et $\Delta T$ tel que $N \Delta T = T$ et notons $\psi_j$ l'approximation de $\psi(j\Delta T)$, $0 \leq j \leq N$.
\subsection{Méthode du splitting d’opérateur}
Considerons l’equation differentielle ordinaire scalaire
\begin{equation} \label{edo1}
\dot{x}=(a+b)x, \quad \quad x(0)=x_0
\end{equation}
ou $a$ et $b$ sont des scalaires. On connaıt la solution exacte de cette equation :
\begin{align*}
x(t)=exp((a+b)t)x_0 &= exp(at)exp(bt)x_0 \quad \mbox{(methode 1)}\\
&= exp(bt)exp(at)x_0 \quad \mbox{(methode 2)}
\end{align*}
Nous pouvons ainsi separer l'evolution selon \eqref{edo1} en deux temps :
$$\mbox{(L1)}
\begin{cases}
\dot{y}=by,\quad y(0)=x_0\\
\dot{x}=ax,\quad x(0)=y(t)
\end{cases}
$$
$$\mbox{(L2)}
\begin{cases}
\dot{y}=ay,\quad y(0)=x_0\\
\dot{x}=bx,\quad x(0)=y(t)
\end{cases}
$$
Pour le systeme (L1), on a clairement
\begin{align*}
x(t) &= exp(at)x_0\\
&= exp(at)y(t)\\
&= exp(at)exp(bt)y(0)\\
&= exp(at)exp(bt)x_0
\end{align*}
Pour (L2), le calcul se fait de la meme maniere et donne le meme resultat.\\\\
On appelle \textbf{splitting de Lie} ou methode a pas fractionnaire les deux methodes (L1) et (L2). Pour une equation scalaire, ces deux methodes sont identiques et reviennent au meme que de traiter l’equation en une seule fois.
Pour exprimer le splitting dans notre exemple, nous avons ete obliges d’introduire une variable intermediaire $y$. Ceci s’avererait a l’usage peu pratique pour des contextes plus compliques. Nous introduisons donc les semi-groupes.
\begin{definition}
	semi groupe Khalil
\end{definition}
\subsubsection{Splittings de Strang}
L’application qui a $x_0$ associe $x(t)$par le flot de l’EDO est le semi-groupe que nous noterons $S(t)$
\begin{equation}
x(t)=S(t)x_0=exp((a+b)t)x_0
\end{equation}
On note $A(t)$ et $B(t)$ les semi-groupes associes aux deux parties de l’equation, a savoir
$$
A(t)x_0=exp(at)x_0, \quad \quad B(t)x_0=exp(bt)x_0
$$
Les deux splittings de Lie (L1) et (L2) consistent donc a ecrire
$$
\mbox{(L1) } x(t)=A(t)B(t)x_0, \quad \quad \mbox{(L2) } x(t)=B(t)A(t)x_0
$$
On definit sans effort deux nouveaux types de splitting : les \textbf{splittings de Strang} \cite{Strang}
$$
\mbox{(S1) } x(t)=A(\dfrac{t}{2})B(t)A(\dfrac{t}{2})x_0, \quad \quad \mbox{(S2) } x(t)=B(\dfrac{t}{2})A(t)B(\dfrac{t}{2})x_0
$$
Les methodes de Lie et de Strang sont respectivement d’ordre 1 et 2.
\subsubsection{Splitting de Strang pour l'équation de Schrödinger}
Afin de presenter le schema de splitting, on considere un probleme d'evolution general. Soit $A$ et $B$, deux operateurs auto-adjoints, tels que : $D(A) \subset \mathcal{H}$, $D(B) \subset \mathcal{H}$ et $A+B$ est un operateur auto-adjoint sur $D(A) \cap D(B)$. On note ici $D(A)$ et $D(B)$ les domaines respectifs des operateurs $A$ et $B$. Considerons le probleme d'evolution
\begin{equation}
\begin{cases}
i \partial_t \psi (x,t) &= A\psi (x,t)+B\psi (x,t), \quad x \in \Omega, t>0 \\
\psi (x,t=0) &= \psi_0(x) \in \mathcal{H}
\end{cases}
\end{equation}
et notons $\psi(x,t)=e^{-i(A+B)t}\psi_0(x)$ sa solution pour $t>0$, et $x\in \Omega$. Le schema de splitting consiste a approcher la solution $\psi$ du probleme d'evolution via une approximation de l'operateur $e^{-i(A+B)t}$ a travers les operateurs $e^{-iA}$ et $e^{-iA}$. Ceci permet alors d'avoir a resoudre successivement deux equations plus simples. 
\\Ici, $A=H_0$ et $B=V(x)-\mu(x)\varepsilon(t)$ est la partie potentielle. On obtient l'approximation
\begin{equation}
\psi(x,t+\Delta T) \approx e^{-iH_0 \frac{\Delta T}{2}}e^{-i(V(x)-\mu\varepsilon)\Delta T}e^{-iH_0 \frac{\Delta T}{2}} \psi(x,t)
\end{equation}
\begin{pro}
	Cette méthode conserve la norme $L^2$ de la fonction d’onde $\psi$ au cours du temps.
\end{pro}
\begin{ proof }
	Les opérateurs exponentiés étant anti-hermitiens, leurs exponentielles sont en effet unitaires.
\end{ proof }
Cette propriété importante \eqref{normecons} est donc conservée après discrétisation.\\
D’autres méthodes de résolution approchée peuvent être envisagées : schémas d’Euler, Runge-Kutta... Le problème posé par ces schémas est qu’ils ne conservent pas la norme de la solution en général.
\subsection{Schéma de Cranck-Nicholson}
Une exception existe avec le schéma implicite de Cranck-Nicholson défini dans notre cas par :
\begin{equation}
i\dfrac{\psi_{j+1}-\psi_j}{\Delta T}=(H_0-\mu\varepsilon_j)\dfrac{\psi_{j+1}+\psi_j}{2}
\end{equation}
En tant que schéma implicite, son inconvénient majeur est d’être coûteux.
\chapter{Chap2}

\section{Sec 2.1}

\chapter{Chap3}

\section{Sec 3.1}

\chapter*{Conclusion et perspectives}\addcontentsline{toc}{chapter}{Conclusion et perspectives}

Le contrôle de l’équation de Schrödinger fait partie d’un domaine de recherche plus vaste appelé contrôle quantique. Le contrôle quantique a connu une activité intense ces deux dernières décennies; ce n'est pas surprenant compte tenu de l'importance de la modélisation quantique dans science des matériaux, conception de médicaments, biologie moléculaire, etc.\\
Dans ce mémoire, \\
Nous avons donc \\
Somme toute,\\

Dans un travail futur, nous pensons nous interesser à paralléliser sur plusieurs processeurs l'algorithme 
\addcontentsline{toc}{chapter}{{Annexe A}}
\lhead{\textsl{Annexe A}}
\chapter*{Annexe A: Espaces hermitiens complexes et opérateurs hermitiens}
\section{Espaces hermitiens}
Dans tout ce chapitre E désigne un espace vectoriel sur $\C$. Les conventions qui suivent imposent un choix de l'argument qui est linéaire. Le choix ci-dessous (forme sesquilinéaire à gauche : première variable semi-linéaire, deuxième variable linéaire) est utilisé par les physiciens, ceci étant dû à l'origine à l'utilisation de la notation bra-ket introduite par Dirac \footnote{Paul Dirac, Les Principes de la mécanique quantique [« The Principles of Quantum Mechanics »] (1re éd. 1930) }, mais le choix opposé est courant en mathématiques.
\begin{definition}
Une forme sesquilinéaire à gauche est une application $b: E \times E \mapsto \C$ telle que
\begin{itemize}
	\item pour tout $x \in E$, $b_1: y \mapsto b(x,y)$ est linéaire;
	\item pour tout $y \in E$, $b_2: x \mapsto b(x,y)$ est semi-linéaire:
	\begin{itemize}
		\item[*] pour tous $x_1$, $x_2 \in E$, on a $b_2(x_1+x_2)=b_2(x_1)+b_2(x_2)$
		\item[*] pour tous $x \in E$ et $\lambda \in \C$, on a $b_2(\lambda x)=\bar{\lambda} b_2(x)$
	\end{itemize}
\end{itemize}
Une telle forme est dite hermitienne (ou à symétrie hermitienne) si pour tous $x$, $y \in E$, on a $b(y,x)=\bar{b(x,y)}$
\end{definition}
\begin{remark}
	Remarquons que pour une forme hermitienne, on a $b(x,x)=\overline{b(x,x)}$ donc $b(x,x) \in \R$
\end{remark}
\begin{remark}
	Une application $f : E \times E \rightarrow \C$ est une forme sesquilinéaire à droite, si et seulement si, l'application $b : E \times E \rightarrow \C$, définie par $b (x, y) = f (y, x)$ est sesquilinéaire à gauche. 
\end{remark}
On note $q(x)=b(x,x)$ et on dit que $q$ est la forme quadratique associée a $b$. On a la $q(\lambda x)= |\lambda|^2 q(x)$ pour $\lambda \in \C$.\\

On peut retrouver $b$ à partir de $q$ avec la formule de polarisation
\begin{equation}
b(x,y)=\frac{1}{4}(q(x+y)-q(x-y)+iq(x+iy)-iq(x-iy))
\end{equation}
\begin{definition}
	Un produit scalaire hermitien est une forme sesquilinéaire définie positive, c’est-à-dire qu’on a $q(x) \geq 0$ pour tout $x \in E$ et $q(x)=0$ seulement si $x=0$.
\end{definition}
\begin{definition}
	Un espace hermitien est un espace vectoriel sur $\C$ muni d'un produit scalaire hermitien. On note le produit scalaire $b(x,y)=\langle x, y \rangle$
\end{definition}
\begin{ex}
	L'espace $\C^n$ avec
	\begin{equation*}
	b(u,v)=\sum_{i=1}^{n} \overline{u_i}v_i
	\end{equation*}
\end{ex}
\begin{ex}
	L'espace $L^2(I)$ avec $I \subset \R$
	\begin{equation*}
	b(g,h)=\int_{I} \overline{g(t)}h(t)dt
	\end{equation*}
\end{ex}

\section{Opérateur hermitien}
\begin{definition}
	Soit $u$ un opérateur sur un espace hermitien $E$, on appelle adjoint de $u$, que l'on note $u^*$, l'endomorphisme tel que, pour tous $x$, $y \in E$,
	\begin{equation*}
	\langle u(x), y \rangle = \langle x, u^*(y)\rangle
	\end{equation*}
	
	On dit que $u$ est auto-adjoint ou hermitien si $u=u^*$, antihermitien si $u=-u^*$ et unitaire si $uu^*=Id$.\\
	Si l'opérateur $a$ est borné, alors l'adjoint l'est aussi et sa norme est égale à celle de $a$.
\end{definition}
(definitions de norme d'un operateur)
(operateur borne, continu)
\begin{equation*}
\lVert uu^* \rVert=\lVert u\rVert^2
\end{equation*}
\addcontentsline{toc}{chapter}{{Annexe B}}
\lhead{\textsl{Annexe B}}
\chapter*{Annexe B: Semi-groupes}
\begin{definition}
	Soit $X$ un espace de Banach. Une famille $(T(t))_{t \geq 0}$ d'opérateurs linéaires bornés de $X$ dans $X$ est un semi-groupe fortement continu d'opérateurs linéaires bornés si
	\begin{enumerate}
		\item[(i)] $T(0)=id_X$;
		\item[(ii)] $T(t+s)=T(t)T(s), \: \forall t, s \geq 0$;
		\item[(iii)] $\forall x \in X, \:\: \R \ni t \mapsto T(t)x \in X$ est continue en $0$.  
	\end{enumerate}
\end{definition}
\begin{remark}
	Un semi-groupe d'opérateurs linéaires bornés $(T(t))_{t\geq 0}$, est uniformément continu si
	$$
	\lim_{t \rightarrow 0} ||T(t)-I||=0
	$$
\end{remark}
Un semi-groupe fortement continu d'opérateurs linéaires bornés sur $X$ sera appelé semi-groupe de classe $C_0$ ou simplement $C_0$ semi-groupe.\\

Si seulement (i) et (ii) sont satisfaits, on dit que la famille $(T(t))_{t\geq 0}$ est un semi-groupe.
\begin{definition}
	L'opérateur linéaire $A$ défini par
	\begin{equation}
	D(A)= \left \{ x \in X : \lim_{t \rightarrow 0^+} \dfrac{T(t)x-x}{t} \mbox{ existe} \right \}
	\end{equation}
	et
	\begin{equation}
	Ax=\lim_{t \rightarrow 0^+} \dfrac{T(t)x-x}{t} \quad \mbox{pour } x \in D(A)
	\end{equation}
	est le générateur infinitésimal du semi-groupe $(T(t))_{t\geq 0}$; $D(A)$ est le domaine de $A$
\end{definition}
%\chapter*{{Bibliographie}}\addcontentsline{toc}{chapter}{{Bibliographie}}


\begin{thebibliography}{KNU90}\addcontentsline{toc}{chapter}{{Bibliographie}}
\bibitem{Tannor} Tannor, D., Kazakov, V., Orlov, V.
\emph{Control of photochemical branching: Novel procedures for finding optimal pulses and global upper bounds}. Time Dependent Quantum Molecular Dynamics, edited by Broeckhove J. and Lathouwers L. Plenum, 347–360 (1992)
\vspace{0.5cm}
\bibitem{Zhu} Zhu, W., Rabitz, H.
\emph{A rapid monotonically convergent iteration algorithm for quantum optimal control over the expectation value of a positive definite operator}. J. Chem. Phys. 109, 385–391 (1998)
\vspace{0.5cm}
\bibitem{Maday} Maday, Y., Turinici, G. 
\emph{New formulations of monotonically convergent quantum control algorithms}. J. Chem. Phys. 118, 8191–8196 (2003)
\vspace{0.5cm}
\bibitem{Salomon} Maday, Y., Salomon, J. and Turinici, G..
\emph{Monotonic time-discretized schemes in quantum control}. Num. Math., 2005.
\vspace{0.5cm}
\bibitem{These} Salomon, J.
\emph{Contrôle en chimie quantique : conception et analyse de schémas d’optimisation}. 2005.
\vspace{0.5cm}
\bibitem{Strang} Strang, G. 
\emph{Accurate partial difference methods I: Linear Cauchy problems.}. Arch. Rat. Mech. and An. 12, 392–402 (1963)
\vspace{0.5cm}
\bibitem{Trelat} Trélat, E.
\emph{Contrôle optimal : théorie et applications}. avril 2016.
\vspace{0.5cm}
\bibitem{Dossa} Dossa, A.
\emph{Cours de Physique Quantique}. 2015-2016.
\end{thebibliography}
\end{document}
