\addcontentsline{toc}{chapter}{{Annexe A}}
\lhead{\textsl{Annexe A}}
\chapter*{Annexe A: Espaces de Hilbert complexes et opérateurs hermitiens}
\section{Rappels sur les espaces de Hilbert}
\subsection{Produit scalaire hermitien}
Les conventions qui suivent imposent un choix de l'argument qui est linéaire. Le choix ci-dessous (forme sesquilinéaire à gauche : première variable semi-linéaire, deuxième variable linéaire) est utilisé par les physiciens, ceci étant dû à l'origine à l'utilisation de la notation bra-ket introduite par Dirac \footnote{Paul Dirac, Les Principes de la mécanique quantique [« The Principles of Quantum Mechanics »] (1re éd. 1930) }, mais le choix opposé est courant en mathématiques.\\
Soit E un espace vectoriel sur $\C$.
\begin{definition}
Une forme sesquilinéaire à gauche est une application $b: E \times E \mapsto \C$ telle que
\begin{itemize}
	\item pour tout $x \in E$, $b_1: y \mapsto b(x,y)$ est linéaire;
	\item pour tout $y \in E$, $b_2: x \mapsto b(x,y)$ est semi-linéaire:
	\begin{itemize}
		\item[*] pour tous $x_1$, $x_2 \in E$, on a $b_2(x_1+x_2)=b_2(x_1)+b_2(x_2)$
		\item[*] pour tous $x \in E$ et $\lambda \in \C$, on a $b_2(\lambda x)=\bar{\lambda} b_2(x)$
	\end{itemize}
\end{itemize}

\end{definition}
\begin{definition}
	Une forme sesquilinéaire est dite hermitienne (ou à symétrie hermitienne) si pour tous $x$, $y \in E$, on a $b(y,x)=\overline{b(x,y)}$
\end{definition}
\begin{remark}
	Remarquons que pour une forme hermitienne, on a $b(x,x)=\overline{b(x,x)}$ donc $b(x,x) \in \R$
\end{remark}
\begin{remark}
	Une application $f : E \times E \rightarrow \C$ est une forme sesquilinéaire à droite, si et seulement si, l'application $b : E \times E \rightarrow \C$, définie par $b (x, y) = f (y, x)$ est sesquilinéaire à gauche. 
\end{remark}
\begin{proposition}
L’ensemble des formes sesquilinéaires hermitiennes sur $E$ est un espace vectoriel complexe.
\end{proposition}
On note $q(x)=b(x,x)$ et on dit que $q$ est la forme quadratique associée a $b$. On a $q(\lambda x)= |\lambda|^2 q(x)$ pour $\lambda \in \C$.

\paragraph*{Identités de polarisation}
$ $\\
Soit $b$ une forme hermitienne. Soit $x$, $y \in E$.
\begin{equation}
b(x+y, x+y)=b(x,x)+b(y,y)+2\Re (b(x,y))
\end{equation}
\begin{equation}
b(x+iy,x+iy)=b(x,x)+b(y,y)-2\Im (b(x,y))
\end{equation}
\begin{equation}
b(x,y)=\frac{1}{4}[b(x+y,x+y)-b(x-y,x-y)+ib(x+iy,x+iy)-ib(x-iy,x-iy)]
\end{equation}
\begin{equation}
\Re (b(x,y))=\frac{1}{4} [b(x+y,x+y)-b(x-y,x-y)]
\end{equation}
\begin{equation}
\Im (b(x,y))=-\frac{1}{4} [b(x+iy,x+iy)-b(x-iy,x-iy)]
\end{equation}
\begin{definition}
	Un produit scalaire hermitien est une forme sesquilinéaire hermitienne définie positive, c’est-à-dire toute application $b$ de $E \times E$ dans $\C$ telle que :
	\begin{enumerate}
		\item $b$ est une forme sesquilinéaire hermitienne
		\item $\forall x \in E, \:\: b(x,x)\geq 0$
		\item $\forall x \in E, \:\: b(x,x)=0 \Rightarrow x=0$
	\end{enumerate}
\end{definition}
\begin{definition}
	Un $\C-$espace vectoriel muni d'un produit scalaire est appelé espace préhilbertien complexe.\\
	Si, de plus, il est de dimension finie, on dira que c’est un espace hermitien. On note le produit scalaire $b(x,y)=\langle x, y \rangle$
\end{definition}
\begin{ex}
	L'espace $\C^n$ avec
	\begin{equation*}
	b(u,v)=\sum_{i=1}^{n} \overline{u_i}v_i
	\end{equation*}
\end{ex}
\begin{ex}
	L'espace $L^2(I)$ avec $I \subset \R$
	\begin{equation*}
	b(g,h)=\int_{I} \overline{g(t)}h(t)dt
	\end{equation*}
\end{ex}
\subsection{Norme hermitienne}
\begin{definition}
	Soit $E$ un espace préhilbertien complexe. On pose $\lVert x \rVert= \sqrt{\langle x,x \rangle}$ On définit ainsi une norme sur
	$E$ dite norme hermitienne
\end{definition}
\paragraph*{Relations entre norme et produit scalaire : identités de polarisation}
$ $\\Soit $E$ un espace préhilbertien complexe. $x$, $y \in E$.
\begin{equation}
||x+y||^2=||x||^2+||y||^2 +2\Re (\langle x,y\rangle)
\end{equation}
\begin{equation}
||x+iy||^2=||x||^2+||y||^2 -2\Im (\langle x,y\rangle)
\end{equation}
\begin{equation}
\langle x,y \rangle=\frac{1}{4}[||x+y||^2-||x-y||^2-i||x+iy||^2-i||x-iy||^2]
\end{equation}
\begin{equation}
||x+y||^2+||x-y||^2=2(|||x||^2+|y||^2)
\end{equation}
\begin{definition}
	Un espace préhilbertien complet pour la norme associée au produit scalaire est
	appelé espace de Hilbert.
\end{definition}
\section{Opérateur hermitien}
Dans toute cette section, $H_1$ et $H_2$ désignent deux espaces de Hilbert. On dira que $T : H_1 \rightarrow
H_2$ est un opérateur borné de $H_1$ dans $H_2$ si $T$ est une application linéaire continue de $H_1$
dans $H_2$. La définition ci-dessous généralise aux espaces de Hilbert la notion d’endomorphisme adjoint vue en L2 dans le cadre des espaces euclidiens.
\begin{theorem}
Soit $T$ un opérateur borné de $H_1$ dans $H_2$. Il existe un et un seul opérateur borné $T^*$ de $H_2$ dans $H_1$ vérifiant
\begin{equation*}
\forall (f,g) \in H_1 \times H_2, \langle Tf,g \rangle_{H_2} = \langle f,T^* g \rangle_{H_1}
\end{equation*}
L’opérateur $T^*$ est appelé opérateur adjoint de $T$, et l'on a $||T||_{\mathcal{L}(H_1,H_2)} = ||T^*||_{\mathcal{L}(H_2,H_1)}$
\end{theorem}
\begin{definition}
	On dit que $T\in \mathcal{L}(H)$ est un opérateur auto-adjoint ou hermitien si$T=T^*$, c’est-à-dire : pour tous $x$, $y \in H$,
	\begin{equation*}
	\langle Tx, y \rangle = \langle x, Ty \rangle
	\end{equation*} Antihermitien si $A=-A^*$.\\
	Unitaire si $AA^*=Id$.
\end{definition}
Si $T$ est auto-adjoint alors pour tout $x \in H$, $\langle Tx , x\rangle$ est réel (même si $H$ est un espace de Hilbert complexe).\\
Comme en dimension finie avec les endomorphismes symétriques, on peut définir les notions de positivité et de négativité pour les opérateurs auto-adjoints.
\begin{definition}
On dit qu’un opérateur auto-adjoint $T$ est positif si\begin{equation*}
\forall x \in H, \langle Tx,x \rangle \geq 0
\end{equation*}
\end{definition}
On dit qu’il est défini positif si l’inégalité ci-dessus est stricte pour $x \neq 0$.\\
On dit que $T$ est négatif (resp. défini négatif) si $-T$ est positif (resp. défini positif).
\begin{proposition}
Si $T$ est opérateur auto-adjoint alors toutes ses valeurs propres sont réelles. Si de plus il est positif (resp. défini positif) alors toutes ses valeurs propres sont positives (resp. strictement positives).
\end{proposition}
