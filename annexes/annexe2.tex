\addcontentsline{toc}{chapter}{{Annexe B}}
\lhead{\textsl{Annexe B}}
\chapter*{Annexe B: Semi-groupes}
\begin{definition}
	Soit $X$ un espace de Banach. Une famille $(T(t))_{t \geq 0}$ d'opérateurs linéaires bornés de $X$ dans $X$ est un semi-groupe fortement continu d'opérateurs linéaires bornés si
	\begin{enumerate}
		\item[(i)] $T(0)=id_X$;
		\item[(ii)] $T(t+s)=T(t)T(s), \: \forall t, s \geq 0$;
		\item[(iii)] $\forall x \in X, \:\: \R \ni t \mapsto T(t)x \in X$ est continue en $0$.  
	\end{enumerate}
\end{definition}
\begin{remark}
	Un semi-groupe d'opérateurs linéaires bornés $(T(t))_{t\geq 0}$, est uniformément continu si
	$$
	\lim_{t \rightarrow 0} ||T(t)-I||=0
	$$
\end{remark}
Un semi-groupe fortement continu d'opérateurs linéaires bornés sur $X$ sera appelé semi-groupe de classe $C_0$ ou simplement $C_0$ semi-groupe.\\

Si seulement (i) et (ii) sont satisfaits, on dit que la famille $(T(t))_{t\geq 0}$ est un semi-groupe.
\begin{definition}
	L'opérateur linéaire $A$ défini par
	\begin{equation}
	D(A)= \left \{ x \in X : \lim_{t \rightarrow 0^+} \dfrac{T(t)x-x}{t} \mbox{ existe} \right \}
	\end{equation}
	et
	\begin{equation}
	Ax=\lim_{t \rightarrow 0^+} \dfrac{T(t)x-x}{t} \quad \mbox{pour } x \in D(A)
	\end{equation}
	est le générateur infinitésimal du semi-groupe $(T(t))_{t\geq 0}$; $D(A)$ est le domaine de $A$
\end{definition}