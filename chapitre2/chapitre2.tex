\chapter{Schémas monotones pour l'équation de Schrödinger}

\section{Introduction}

Considérons un système quantique décrit par une fonction d’onde $\psi(x, t)$ soumis à un champ électrique $\varepsilon(t)$ et dont l’état initial $\psi(x, 0) = \psi_0 (x)$ est supposé connu. Comme stipulé au \eqref{schcon2}, l’évolution d’un tel système est régie par l’équation:

\begin{equation}
\begin{cases}
i \frac{\partial}{\partial t} \psi (x,t) = H\psi(x,t) - \mu(x)\varepsilon(t)\psi(x,t)\\
\psi(x,t=0)=\psi_0(x)
\end{cases}
\end{equation}

Le moment dipolaire $\mu$ est supposé connu et indépendant du temps.

\section{Fonctionnelles de coût}
Soit $\alpha \in \R^+$, $O$ un opérateur symétrique borné, $T$ un réel positif, $\varepsilon$ une fonction de $L^2 ([0, T ], \R),\  \psi : \Omega \times [0, T ] \longrightarrow \C$ une fonction d’onde dépendant du temps et $\psi_{cible} : \Omega \longrightarrow C$ une fonction d’onde fixée.
\\Nous considerons deux fonctionnelles de coût:

\begin{equation}
J_1(\varepsilon) = \langle \psi(T)|O|\psi(T) \rangle - \alpha \int_0^T \varepsilon^2(t)dt
\end{equation}

\begin{equation}
J_2(\varepsilon) = 2\Re\langle \psi_{cible}|\psi(T)\rangle - \alpha \int_0^T \varepsilon^2(t)dt
\end{equation}

$\psi_{cible}$ représente une fonction d’onde cible. $J_1$ releve du formalisme attaché aux observables tandis que $J_2$ est plus simple a manipuler.
%Les deux permettent de réaliser un compromis entre la recherche d'une grande valeur de l’observable et une norme $L_2$ du champ $\varepsilon$ raisonnable.
Le maximum du terme $2\Re\langle \psi_{cible}|\psi(T)\rangle$ est bien atteint pour $\psi(T) = \psi_{cible}$ puisque $\lVert \psi_{cible} \rVert = 1$ et que $2\Re\langle \psi_{cible}|\psi(T)\rangle = -\lVert \psi_{cible} - \psi(T)\rVert ^2+2$.\\

les méthodes numérique d’optimisation reposent sur la résolution des équations d’Euler-Lagrange de la fonctionnelle traitée. Les équations d'Euler-Lagrange sont alors pour $J_1$ :\\\\

\begin{equation}
\begin{cases}
i \frac{\partial}{\partial t} \psi (x,t) = H - \varepsilon(t)\mu(x)\psi(x,t)\\
\psi(x,t=0)=\psi_0(x)
\end{cases}
\end{equation}

\begin{equation}
\begin{cases}
i \frac{\partial}{\partial t} \chi (x,t) = H - \varepsilon(t)\mu(x)\chi(x,t)\\
\chi(x,t=T)=O\psi(x,T)
\end{cases}
\end{equation}

\begin{equation}
\alpha \varepsilon(t) = -\Im \langle \chi(t)|\mu|\psi(t)\rangle 
\end{equation}

et pour $J_2$ :

\begin{equation}
\begin{cases}
i \frac{\partial}{\partial t} \psi (x,t) = H - \varepsilon(t)\mu(x)\psi(x,t)\\
\psi(x,t=0)=\psi_0(x)
\end{cases}
\end{equation}

\begin{equation}
\begin{cases}
i \frac{\partial}{\partial t} \chi (x,t) = H - \varepsilon(t)\mu(x)\chi(x,t)\\
\chi(x,t=T)=\psi_{cible}(x)
\end{cases}
\end{equation}

\begin{equation}
\alpha \varepsilon(t) = -\Im \langle \chi(t)|\mu|\psi(t)\rangle 
\end{equation}

L'existence de maxima de $J_1$ et de $J_2$ et de fonctionnelles plus générales a été montrée dans \cite{Cances} et \cite{Baudouin}.

\section{Schémas monotones}
Les schémas monotones ont été introduits dans le cadre du contrôle quantique par D. Tannor dans \cite{Tannor} et par W. Zhu et H. Rabitz dans \cite{Zhu} sous deux formes différentes. En 2003, Y. Maday et G. Turinici \cite{Maday} présentent une classe plus large de schémas monotones qui englobe les deux types d’algorithmes initiaux.
\begin{equation} \label{maday1}
\begin{cases}
i \frac{\partial }{\partial t} \psi^k (x,t) = (H(x)-\mu(x)\varepsilon^k(t))\psi^k (x,t)\\
\psi (x,t=0) = \psi_0(x)\\
\end{cases}
\end{equation}

\begin{equation} \label{maday2}
\varepsilon^k (t) = (1-\delta)\tilde{\varepsilon}^{k-1}(t)-\frac{\delta}{\alpha} \Im\langle\chi^{k-1}|\mu|\psi^{k}\rangle(t)
\end{equation}

\begin{equation} \label{maday3}
\begin{cases}
i \frac{\partial }{\partial t} \chi^k (x,t) = (H(x)-\mu(x)\tilde{\varepsilon}^k (t))\chi^k (x,t)\\
\chi^k (x,t=T) = O\psi^k(x,T)\\
\end{cases}
\end{equation}

\begin{equation} \label{maday4}
\tilde{\varepsilon}^k (t) = (1-\eta)\varepsilon^{k}(t)-\frac{\eta}{\alpha} \Im\langle\chi^{k}|\mu|\psi^{k}\rangle(t)
\end{equation}
où $\delta$ et $\eta$ sont deux parametres reels. Le schéma Tannor \cite{Tannor} correspond au cas où $(\delta,\eta)= (1, 0)$ tandis que celui de Zhu et Rabitz \cite{Zhu} correspond au cas où $(\delta,\eta)= (1, 1)$.\\\\
La propriété la plus importante de cet algorithme est donnée dans le théorème suivant \cite{Maday}        
\begin{theorem}[Maday, Y., Turinici, G. (2003)]
Supposons que $O$ soit un opérateur semi-défini positif auto-adjoint. Alors, pour tout $\eta$, $\delta$ $\in [0,2]$ l'algorithme donné par les équations \eqref{maday1} à \eqref{maday4} converge de façon monotone dans le sens où $J(\varepsilon^{k+1}) \geq J(\varepsilon^{k})$.
\end{theorem}

\begin{ proof }
Evaluons la difference entre deux valeur de la fonctionnelle $J_1$ entre deux iterations successives. Supposons que $\eta \neq 0$, $\delta \neq 0$. Alors,
\begin{align*}
J(\varepsilon^{k+1})-J(\varepsilon^{k}) &= \langle \psi^{k+1}(T)|O| \psi^{k+1}(T) \rangle-\alpha \int_{0}^{T} \varepsilon^{k+1}(t)^2 dt - \langle \psi^{k}(T)|O| \psi^{k}(T) \rangle\\
&\quad + \alpha \int_{0}^{T} \varepsilon^{k}(t)^2 dt \\
&=\langle \psi^{k+1}(T)-\psi^{k}(T)|O| \psi^{k+1}(T)-\psi^{k}(T)\rangle \\
&\quad +2\Re \langle \psi^{k+1}(T)-\psi^{k}(T)|O|\psi^{k}(T)\rangle\\ 
&\quad +\alpha \int_{0}^{T} \varepsilon^{k}(t)^2 dt-\alpha\int_{0}^{T} \varepsilon^{k+1}(t)^2 dt
\end{align*}
Puisque, nous avons aussi
\begin{align*}
2\Re \langle \psi^{k+1}(T)-\psi^{k}(T)|O|\psi^{k}(T)\rangle &= 2\Re \langle \psi^{k+1}(T)-\psi^{k}(T), O\psi^{k}(T)\rangle\\
&= 2\Re \langle \psi^{k+1}(T)-\psi^{k}(T), \chi^{k}(T)\rangle\\
&= 2\Re \int_{0}^{T} \langle \frac{\partial (\psi^{k+1}(t)-\psi^{k}(t))}{\partial_{t}}, \chi^{k}(t) \rangle\\
&\quad + \langle \psi^{k+1}(t)-\psi^{k}(t), \frac{\partial \chi^{k}(t)}{\partial_{t}} \rangle dt\\
&= 2\Re \int_{0}^{T} \langle \frac{H_0-\mu \varepsilon^{k+1}}{i}\psi^{k+1}(t) - \frac{H_0-\mu \varepsilon^{k}}{i}\psi^{k}(t), \chi^{k}(t) \rangle\\
&\quad + \langle \psi^{k+1}(t)-\psi^{k}(t), \frac{H_0-\mu \tilde{\varepsilon}^k}{i}\chi^{k}(t) \rangle dt\\
&= 2\Re \int_{0}^{T} \varepsilon^{k+1} \langle \frac{-\mu}{i}\psi^{k+1}(t), \chi^{k}(t)\rangle- \varepsilon^{k} \langle \frac{-\mu}{i}\psi^{k}(t), \chi^{k}(t)\rangle\\
&\quad + \tilde{\varepsilon}^{k} \langle \psi^{k+1}(t)-\psi^{k}(t), \frac{-\mu}{i} \chi^{k}(t)\rangle dt\\
&= 2 \int_{0}^{T} \varepsilon^{k+1} \cdot \frac{\alpha (\varepsilon^{k+1}-(1-\delta)\tilde{\epsilon}^k)}{\delta}\\
&\quad - \varepsilon^{k} \cdot \frac{\alpha (\tilde{\varepsilon}^{k}-(1-\eta)\epsilon^k)}{\eta}\\
&\quad - \tilde{\varepsilon}^{k} \cdot \frac{\alpha (\varepsilon^{k+1}-(1-\delta)\tilde{\epsilon}^k)}{\delta}\\
&\quad + \tilde{\varepsilon}^{k+1} \cdot \frac{\alpha (\tilde{\varepsilon}^{k}-(1-\eta)\epsilon^k)}{\eta}dt\\
\end{align*}
nous obtenons donc
\begin{align*}
J(\varepsilon^{k+1})-J(\varepsilon^{k}) &= \langle \psi^{k+1}(T) - \psi^{k}(T)|O| \psi^{k+1}(T) - \psi^{k}(T) \rangle\\
& \quad + \alpha \int_{0}^{T} (\frac{2}{\delta}-1)(\varepsilon^{k+1}-\tilde{\varepsilon}^k)^2 + (\frac{2}{\eta}-1)(\tilde{\varepsilon}^k-\varepsilon^{k})^2 dt
\end{align*}
qui est positif sous les hypotheses du theoreme. Notons que le cas $\delta=0$ entraine $\varepsilon^{k+1}=\tilde{\varepsilon}^k$ et $\eta=0$ entraine $\tilde{\varepsilon}^{k}=\varepsilon^k$ et la meme conclusion est obtenue.
\end{ proof }
