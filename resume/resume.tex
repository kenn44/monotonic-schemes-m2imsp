\chapter*{Résumé \& Abstract}\addcontentsline{toc}{chapter}{Résumé \& Abstract}
\Large
\begin{flushleft}
\textbf{\underline{Résumé}}
\end{flushleft}
\normalsize
Ce mémoire traite de l'étude de quelques méthodes d'adaptation de maillage dans le cadre de la méthode des éléments finis. Partant d'une équation aux dérivées partielles à résoudre numériquement, nous cherchons le maillage le plus adapté possible afin d'obtenir une approximation précise de la solution éléments finis. L'adaptation par métrique et l'adaptation hiérarchique sont les méthodes abordées. Les bases et les différentes particularités de ces méthodes sont également exposées. Nous introduisons ensuite une nouvelle méthode d'adaptation utilisant les statistiques et les avantages du calcul parallèle dans l'adaptation de maillage. La démarche suivie est basée sur la conduite de l'étude d'une famille de maillage générée sur différents processeurs et possédant un certain nombre de caractéristiques statistiques données. Nous voulons utiliser la dispersion des interpolés sur cette famille de maillage comme indicateur d'erreur.\\\\
\Large\textbf{\underline{Mots-clés}}\normalsize\\\\
Adaptation de maillage, calcul parallèle, écart type, équation aux dérivées partielles, maillage, méthode des éléments finis, moyenne.
\[\]\[\]\[\]
\Large\textbf{\underline{Abstract}}\\\\\normalsize
This dissertation is devoted to the study of some mesh adaptation methods for the finite element method. Given a partial differential equation to be solved numerically, we seek the most suitable mesh to obtain an accurate approximation of the solution finite elements. The adaptation methods discussed are metric adaptation and hierarchical adaptation. We present in this dissertation the bases and the characteristics of these two methods. We then introduce a new adaptation method using the statistics and advantages of parallel computation in the mesh adaptation. The approach followed is based on the conduct of the study of a family of mesh generated on differents processors and having a certain number of statistical characteristics given. We want to use the interpolated dispersion on this mesh family as error indicator.\\\\
\Large\textbf{\underline{Key words}}\normalsize\\\\
Mesh adaptation, parallel computation, standard deviation, partial differential equation, mesh, finite elements method, mean.