\chapter*{Résumé \& Abstract}\addcontentsline{toc}{chapter}{Résumé \& Abstract}
\Large
\begin{flushleft}
\textbf{\underline{Résumé}}
\end{flushleft}
\normalsize
Les problèmes de contrôle optimal sur les systèmes quantiques suscitent un vif intérêt, aussi bien pour les questions fondamentales que pour les applications existantes et futures. Un problème important est le développement de méthodes de construction de contrôles pour les systèmes quantiques. Une des méthodes couramment utilisée est la méthode de Krotov initialement proposée dans un cadre plus général dans les articles de V.F. Krotov et I.N. Feldman (1978 \cite{Krotov1}, 1983 \cite{Krotov2}). Cette méthode a été utilisée pour développer une nouvelle approche permettant de determiner des contrôles optimaux pour les systèmes quantiques dans \cite{Tannor} et dans de nombreux autres travaux de recherche: \cite{Zhu}, \cite{Maday} et \cite{Salomon} notamment. Leur mise en œuvre numérique repose souvent sur des discrétisations liées à des développement limités. Cette approche entraîne cependant parfois des instabilités numériques. Nous présentons ici plusieurs méthodes de discrétisation temporelle qui permettent de résoudre ce problème en conservant au niveau discret la monotonie des schémas.\\\\
\textbf{Mots-clés}: contrôle quantique, schémas monotones convergents, methode de Krotov
\[\]
\Large\textbf{\underline{Abstract}}\\\\\normalsize
Mathematical problems of optimal control in quantum systems attract high interest in connection with fundamental questions and existing and prospective applications. An important problem is the development of methods for constructing controls for quantum systems. One of the commonly used methods is the Krotov method initially proposed beyond quantum control in the articles by V.F. Krotov and I.N. Feldman (1978 \cite{Krotov1}, 1983 \cite{Krotov2}). The method was used to develop a novel approach for finding optimal controls for quantum systems in \cite{Tannor}, and in many works of various scientists: \cite{Zhu}, \cite{Maday} and \cite{Salomon} especially. However, the properties of the discrete version of these procedures have not been yet tackled with.
We present here a stable time and space discretization which preserves the monotonic properties of the monotonic algorithms.\\\\
\textbf{Keywords}: quantum control, monotonically convergent algorithms, Krotov method
