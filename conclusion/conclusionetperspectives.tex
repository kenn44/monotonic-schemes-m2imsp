\chapter*{Conclusion et perspectives}\addcontentsline{toc}{chapter}{Conclusion et perspectives}

Le contrôle de l’équation de Schrödinger fait partie d’un domaine de recherche plus vaste appelé contrôle quantique. Le contrôle quantique a connu une activité intense ces deux dernières décennies; ce n'est pas surprenant compte tenu de l'importance de la modélisation quantique dans la science des matériaux, conception de médicaments, biologie moléculaire, etc.\\
Dans ce mémoire, nous nous sommes intéressés à l'étude d'une classe d'algorithme particulièrement adaptée aux problèmes de contrôle quantique : les schémas monotones.\\
Nous avons donc étudié ces algorithmes puis construit différentes discrétisations qui préservent la propriété de monotonie.\\
Somme toute, les simulations numériques effectuées montrent que les algorithmes construits sont bien monotones et convergent tres rapidement.\\\\

Dans un travail futur, nous pensons, dans un premier temps intégrer une discrétisation spatiale avec les méthodes de Galerkin. Ensuite, nous nous intéresserons à paralléliser sur plusieurs processeurs ces algorithmes. Enfin, nous voulons également étudier d'autre problèmes en contrôle quantique, notamment les condensats de Bose-Einstein et les problèmes relevant de l'informatique quantique.