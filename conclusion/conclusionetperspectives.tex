\chapter*{Conclusion et perspectives}\addcontentsline{toc}{chapter}{Conclusion et perspectives}

Dans ce mémoire, nous nous sommes attachés à exposer les rudiments de l'adaptation de maillage ainsi que les notions nécessaires à sa compréhension. L'adaptation de maillage est un outil intéressant pour la résolution numérique des équations aux dérivées partielles et grâce à cet outil, l'on gagne considérablement dans la précision des résultats obtenus.\\

Nous avons donc abordé deux méthodes d'adaptation à savoir l'adaptation métrique qui associe à chaque arête des éléments, une métrique dépendant de l'erreur commise lors de l'interpolation et l'adaptation hiérarchique qui approche l'erreur sur la solution éléments finis en construisant une nouvelle solution à l'aide d'une base hiérarchique.\\
Somme toute, ces deux méthodes se basent sur des estimations de dérivées puisque la solution exacte au problème n'est pas connue à priori. Pour ce faire, des méthodes d'estimations de dérivées telles que celle décrite par  Zhang, Z. et A. Naga dans~~\cite{2}~~sont utilisées. Nous introduisons dans ce mémoire une méthode d'estimations de dérivées basée sur le comportement d'une famille de maillage possédant des propriétés statistiques données.\\

Dans un travail futur, nous pensons nous interesser à paralléliser sur plusieurs processeurs l'algorithme d'estimation de dérivées afin de réduire le temps de calcul. Nous nous interesserons aussi à trouver théoriquement la constante exhibée dans la partie~~\ref{partieconstante}~~mais aussi à une démonstration théorique de la convergence du processus mis en place.