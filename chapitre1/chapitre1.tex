\chapter{Mécanique quantique et contrôle quantique}

\section{La mécanique quantique en trois postulats}
\subsection{Premier postulat de la mécanique quantique}
\subsubsection{Fonction d'onde}
En mécanique quantique, l'état d'un système donné, est donné par son vecteur d'état $|\psi (t)\rangle$. 
%On utilise la notation bra-ket introduite par Dirac \footnote{Paul Dirac, Les Principes de la mécanique quantique [« The Principles of Quantum Mechanics »] (1re éd. 1930) } pour représenter les états quantiques de manière concise et simple. Le vecteur $\psi$ est ainsi noté $|\psi \rangle$ et appelé \textbf{ket}, tandis que son vecteur dual est appelé \textbf{bra} et noté $\langle \psi |$. \\
L'espace des états dépend du système considéré. Par exemple, dans le cas le plus simple où le système n'a pas de spin, les états quantiques sont des fonctions $\psi$ :
$$
\begin{aligned}
	\psi :\quad \quad \mathbb {R} ^{3}&\rightarrow \mathbb{C} \\(x,y,z)&\mapsto \psi (x,y,z)
\end{aligned}
$$
telles que l'intégrale  $\int _{\mathbb{R} ^{3}}|\psi (\mathbf{r} )|^{2}d\mathbf{r}$ converge. Dans ce cas, $\psi$ est appelée la fonction d'onde du système.
\subsubsection{Cas d’une particule dans l’espace à une dimension}
La probabilité pour que la particule soit dans l’intervalle $[a,b]$ est donnée par l’aire de la courbe située entre $x=a$ et $x=b$
\begin{equation}
\int_a^b dP(x)= \int_a^b |\psi(x,t)|^2dx
\end{equation}
Il est impossible de connaître avec précision la position de la particule à un instant t. On ne peut que connaître la probabilité $dP(x)$ pour qu’elle soit entre $x$ et $x+dx$, soit:
\begin{equation}
dP(x)=|\psi(x,t)|^2dx=\psi(x,t)\overline{\psi(x,t)}dx
\end{equation}
La particule doit être quelque part sur l’axe $X’OX$, par conséquent:
\begin{equation}
\int_{-\infty}^{+\infty}|\psi(x,t)|^2dx=1
\end{equation}
pour tout $t$. $\psi$ est donc de carré sommable.
La densité de probabilité est donnée par
\begin{equation}
\dfrac{dP(x,t)}{dx}=|\psi(x,t)|^2=\rho(x,t)
\end{equation}

\subsubsection{Cas d’une particule dans l’espace à trois dimensions}
On a 
\begin{equation}
\int dP(\vec{r},t)=\iiint_{espace}|\psi(x,t)|^2d^3r=1
\end{equation}

Où $d^3r$ représente l'élément de volume donnée par: $$d^3r=dxdydz=r^2sin\theta dr d\theta d\varphi$$

\subsubsection{Cas de N particules}
L’espace le mieux adapté à la description des systèmes en physique quantique est un espace $\Omega$, nommé espace des configurations qui représente l’ensemble de toutes les configurations possibles du système. Par exemple, dans le cas d’un système à $N$ particules isolées et sans contraintes, l’espace des configurations est $\Omega = \R^{3N}$ et $\psi(x,t) \in L^2(\Omega, \C)$.

\begin{Post}
	A tout système quantique correspond un espace de Hilbert complexe $\mathcal{H}$, tel que l’ensemble des états accessibles au système soit en bijection avec la sphère unité de $\mathcal{H}$.
\end{Post}

\subsection{Observables et deuxième postulat}
Une observable est l'équivalent en mécanique quantique d'une grandeur physique en mécanique classique, comme la position, la quantité de mouvement, l'énergie, etc. Une observable est formalisée mathématiquement par un opérateur hermitien sur $\mathcal{H}$ (chaque état quantique est représenté par un vecteur de $\mathcal{H}$). 
%Cet operateur permet de décomposer un état quantique quelconque $|\psi \rangle$ en une combinaison linéaire d'états propres,
\begin{ex}
	Exemples d'observables
	\begin{enumerate}
		\item la position: $X$
		\item l'énergie potentielle: $V$
		\item la quantité de mouvement: $P=-i\hbar \nabla$
		\item l'énergie cinétique:
		$H_0=\dfrac{P \cdot P}{2m}=-\dfrac{\hbar ^{2}}{2m}\nabla^{2}$
		\item l'énergie totale, appelée hamiltonien:
		$H=H_0+V$
	\end{enumerate}
\end{ex}
%En mécanique quantique, une grandeur ne prend une valeur déterminée que lors d’une mesure:
\begin{Post}
	A toute grandeur physique représentée par l'observable $\mathcal{A}$ correspond un opérateur $A$ auto-adjoint sur $\mathcal{H}$, vérifiant la propriété suivante : le résultat de la mesure d’une grandeur physique $\mathcal{A}$ ne peut être qu’un élément du spectre de $A$.
\end{Post}
Les valeurs propres sont les valeurs pouvant résulter d'une mesure idéale de cette propriété, les vecteurs propres étant l'état quantique du système immédiatement après la mesure et résultant de cette mesure. Ce postulat peut donc s'écrire :
$$
A|\alpha _{n}\rangle =a_{n}|\alpha _{n}\rangle 
$$
où $A$, $|\alpha_{n}\rangle$ et $a_n$ désignent, respectivement, l'observable, le vecteur propre et la valeur propre correspondante.\\
Les états propres de tout observable $A$ forment une base orthonormée dans l'espace de Hilbert $\mathcal{H}$.
Cela signifie que tout vecteur $|\psi (t)\rangle$ peut se décomposer de manière unique sur la base de ces vecteurs propres $|\phi_{i}\rangle$:
$$
|\psi \rangle =c_{1}|\phi _{1}\rangle +c_{2}|\phi _{2}\rangle +...+c_{n}|\phi _{n}\rangle +...
$$
La moyenne des mesures de $\mathcal{A}$ est quant à elle égale à $\langle\psi|A|\psi\rangle$ où la notation $\langle\cdot|A|\cdot\rangle$ est définie par :
\begin{equation}
\langle\psi|A|\chi\rangle = \int_{\Omega} \bar{\psi}A\chi
\end{equation}
%ou $\psi$ et $\chi$ sont des fonctions de $L^2(\Omega, \C)$ et $A$ un opérateur arbitraire défini de $L^2(\Omega, \C)$ dans lui-même.
\begin{ex}
	La position moyenne sur l’axe des abscisses:
	\begin{equation*}
	\langle X \rangle = \int_{- \infty}^{- \infty} x |\psi(x,t)|^2 dx
	\end{equation*}
\end{ex}

\subsection{Equation de Schrödinger et troisième postulat}
L'équation de Schrödinger, proposée par le physicien autrichien Erwin Schrödinger en 1952, est une équation fondamentale en mécanique quantique. Elle décrit l'évolution dans le temps d’une particule massive non relativiste, et remplit ainsi le même rôle que la relation fondamentale de la dynamique en mécanique classique $\dfrac{dp}{dt}=F$.
\begin{Post}
	Entre deux mesures, l’évolution de l’état est régie par l’équation de Schrödinger
	\begin{equation} \label{sch1}
	i\hbar \dfrac{\partial }{\partial t} \psi (x,t) = H\psi (x,t)
	\end{equation}
\end{Post}
%\subsubsection{Exemple de resolution de \eqref{sch1}: cas d'une particule libre}
Considérons le cas d'une particule libre ($V=0$) et menons notre étude sur $\R^m$ $(m \geq 1)$ nous étudions alors l'équation
\begin{equation} \label{chauchy2}
u'(t)=Au(t)
\end{equation}
où $A$ est un opérateur sur $E=L^2(R^m)$ défini par 
$$
Au=ik\Delta u=ik(\partial_{1}^2u+...+\partial_{m}^2u)
$$
$k$ une constante réelle. Le domaine de $A$ est l'ensemble des $u \in L^2(R^m)$ tel que $\Delta u$ (au sens des distributions) appartient à $L^2$.
\begin{definition}
	On dit que le problème de Cauchy pour l'équation \eqref{chauchy2} est bien posé si les deux hypothèses suivantes sont satisfaites:
	\begin{enumerate}
		\item[(a)] \textbf{Existence et unicité de solutions}: il existe un sous-espace dense $D$ de $E$ tel que, pour tout $u_0 \in D$, il existe une unique solution $u(.)$ de \eqref{chauchy2} avec $u(0)=u_0$
		\item[(b)] \textbf{La solution dépend de façon continue des données}: il existe une fonction non décroissante, positive $C(t)$ tel que
		$$
		||u(t)||\leq C(t) ||u(0)||
		$$
	\end{enumerate}
\end{definition}
En appliquant la transformée de Fourier à \eqref{chauchy2}, nous obtenons
\begin{align*} 
\mathcal{F}[Au](\xi) &= \mathcal{F}[ik\Delta u](\xi) \\ 
&= ik\mathcal{F}[\Delta u](\xi) \\
&= -ik|\xi|^2 \mathcal{F}[u](\xi) \\
\mathcal{F}[Au](\xi) &= -ik|\xi|^2 \tilde{u}(\xi)
\end{align*}
Alors si $u_0 \in D(A)$ nous obtenons une solution:
\begin{equation}
u(t)=\mathcal{F}^{-1}(exp(-ik|\xi|^2 t)\mathcal{F}[u_0](\xi))
\end{equation}
Puisque $D(A)$ est dense dans $E$, on déduit que \eqref{chauchy2} admet une unique solution sur $E$.\\
Ensuite, observons que si $u \in D(A)$, alors
\begin{align*}
(Au,u) &= (\mathcal{F}[Au],\mathcal{F}[u])\\
&= ik\int|\xi|^2 |\tilde{u}|^2
\end{align*}
Ainsi, $\Re (Au,u)=0$\\
Par conséquent, si $u(.)$ est une solution de \eqref{chauchy2}, nous avons:
\begin{align*}
\partial_{t}||u(t)||^2 &= 2 \Re (u'(t),u(t))\\
&= 2 (Au(t),u(t))\\
&=0
\end{align*}
Ainsi, $||u(t)||$ est constante:
\begin{equation} \label{normecons}
||u(t)||=||u(0)|| \quad \forall t \in \R
\end{equation}
En conclusion, le problème de Cauchy pour \eqref{chauchy2} est bien posé pour tout $t$.\\
En outre, \cite{Fattorini} généralise ce résultat sur $E=L^p (\R^m)$ avec $1 \leq p < m$.

\section{Contrôle quantique}
On rappelle que l'état $\psi(t)$ d'un système quantique évolue conformément à l'équation de Schrödinger:
\begin{equation}
i \hbar \dfrac{\partial }{\partial t} \psi (x,t) = H\psi (x,t),\quad \quad \psi(t=0)=\psi_0
\end{equation}
où $H$ est un endomorphisme auto-adjoint sur $\mathcal{H}$ appelé Hamiltonien interne du système. $\hbar$ est la constante de Planck. Dans la suite, nous travaillons en unités atomiques, c'est-à-dire $\hbar=1$ et l'état initial a une norme unitaire, $||\psi_0||^2 \equiv \langle \psi_0|\psi_0 \rangle =1$.
\subsection{Exemple de modèle bilinéaire (BLM)}
Pour simplifier, nous considérons un système quantique de dimension finie. C'est une approximation appropriée dans de nombreuses situations pratiques. Pour un système quantique de dimension $N$, $\mathcal{H}$ est un espace de Hilbert complexe de dimension $N$. Les valeurs propres $(\phi_{i})_{1 \leq i \leq N}$ de $H$ forment une base orthogonale de $\mathcal{H}$.\\
Dans de nombreuses situations, le contrôle du système peut être réalisé par un ensemble de fonctions de contrôle $\varepsilon_{k}(t) \in \R$ couplées au système via une famille d'opérateurs hermitiens indépendants du temps $\{H_k, k=1,2,...\}$. L'Hamiltonien total $H+\sum_{k} \varepsilon_{k}(t)H_k$ détermine alors la nouvelle dynamique du système.
\begin{equation} \label{schcon}
i \dfrac{\partial }{\partial t} \psi (x,t) = (H+\sum_{k} \varepsilon_{k}(t)H_k)\psi (x,t)
\end{equation}
Le but typique d'un problème de contrôle quantique défini sur le système \eqref{schcon} est de trouver en temps fini $T > 0$ un ensemble de contrôles admissibles $u_k(t) \in \R$ qui conduit le système d'un état initial $\psi_0$ à un état prescrit $\psi_{cible}$.
\subsection{Cas d'un champs laser}
En vue de modéliser les interactions onde-matière à l'échelle atomique, nous introduisons un contrôle, généré par un dipôle électrostatique de moment dipolaire $\mu (x)$, émettant un champs (électrique) laser, d'amplitude $\varepsilon (t)$ dépendant du temps.\\
La dynamique du système est désormais donnée par:
\begin{equation} \label{schcon2}
\begin{cases}
i \dfrac{\partial }{\partial t} \psi (x,t) &= H\psi (x,t)-\mu(x)\varepsilon(t)\psi (x,t) \\
\psi (x,t=0) &= \psi_0(x)
\end{cases}
\end{equation}

\section{Discrétisation temporelle}
Choisissons deux paramètres de discrétisation temporelle $N$ et $\Delta T$ tels que $N \Delta T = T$ et notons $\psi_j$ l'approximation de $\psi(j\Delta T)$, $0 \leq j \leq N$.
\subsection{Méthode du splitting d’opérateur}
Considérons l’équation différentielle ordinaire scalaire
\begin{equation} \label{edo1}
\dot{x}=(a+b)x, \quad \quad x(0)=x_0
\end{equation}
où $a$ et $b$ sont des scalaires. On connaît la solution exacte de cette equation :
\begin{align*}
x(t)=exp((a+b)t)x_0 &= exp(at)exp(bt)x_0 \quad \mbox{(methode 1)}\\
&= exp(bt)exp(at)x_0 \quad \mbox{(methode 2)}
\end{align*}
Nous pouvons ainsi séparer l'évolution selon \eqref{edo1} en deux temps :
$$\mbox{(L1)}
\begin{cases}
\dot{y}=by,\quad y(0)=x_0\\
\dot{x}=ax,\quad x(0)=y(t)
\end{cases}
$$
$$\mbox{(L2)}
\begin{cases}
\dot{y}=ay,\quad y(0)=x_0\\
\dot{x}=bx,\quad x(0)=y(t)
\end{cases}
$$
Pour le système (L1), on a clairement
\begin{align*}
x(t) &= exp(at)x_0\\
&= exp(at)y(t)\\
&= exp(at)exp(bt)y(0)\\
&= exp(at)exp(bt)x_0
\end{align*}
Pour (L2), le calcul se fait de la même manière et donne le même résultat.\\\\
On appelle \textbf{splitting de Lie} ou méthode à pas fractionnaire les deux méthodes (L1) et (L2). Pour une equation scalaire, ces deux méthodes sont identiques et reviennent au même que de traiter l’équation en une seule fois.
Pour exprimer le splitting dans notre exemple, nous avons été obligés d’introduire une variable intermédiaire $y$. Ceci s’avérerait à l’usage peu pratique pour des contextes plus compliqués. Nous introduisons donc le semi-groupe $S(t)$.

\subsubsection{Splittings de Strang}
L’application qui à $x_0$ associe $x(t)$ par le flot de l’EDO est le semi-groupe que nous noterons $S(t)$
\begin{equation}
x(t)=S(t)x_0=exp((a+b)t)x_0
\end{equation}
On note $A(t)$ et $B(t)$ les semi-groupes associés aux deux parties de l’équation, à savoir
$$
A(t)x_0=exp(at)x_0, \quad \quad B(t)x_0=exp(bt)x_0
$$
Les deux splittings de Lie (L1) et (L2) consistent donc à écrire
$$
\mbox{\textbf{(L1)}: } x(t)=A(t)B(t)x_0, \quad \quad \mbox{\textbf{(L2)}: } x(t)=B(t)A(t)x_0
$$
On definit sans effort deux nouveaux types de splitting : les \textbf{splittings de Strang} \cite{Strang}
$$
\mbox{\textbf{(S1)}: } x(t)=A(\dfrac{t}{2})B(t)A(\dfrac{t}{2})x_0, \quad \quad \mbox{\textbf{(S2)}: } x(t)=B(\dfrac{t}{2})A(t)B(\dfrac{t}{2})x_0
$$
Les méthodes de Lie et de Strang sont respectivement d’ordre 1 et 2.
\subsubsection{Splitting de Strang pour l'équation de Schrödinger}
Afin de présenter le schéma de splitting, on considère un problème d'évolution général. Soit $A$ et $B$, deux opérateurs auto-adjoints, tels que : $D(A) \subset \mathcal{H}$, $D(B) \subset \mathcal{H}$ et $A+B$ est un opérateur auto-adjoint sur $D(A) \cap D(B)$. On note ici $D(A)$ et $D(B)$ les domaines respectifs des opérateurs $A$ et $B$. Considérons le problème d'évolution
\begin{equation}
\begin{cases}
i \partial_t \psi (x,t) &= A\psi (x,t)+B\psi (x,t), \quad x \in \Omega, t>0 \\
\psi (x,t=0) &= \psi_0(x) \in \mathcal{H}
\end{cases}
\end{equation}
et notons $\psi(x,t)=e^{-i(A+B)t}\psi_0(x)$ sa solution pour $t>0$, et $x\in \Omega$. Le schéma de splitting consiste à approcher la solution $\psi$ du problème d'évolution via une approximation de l'opérateur $e^{-i(A+B)t}$ à travers les opérateurs $e^{-iA}$ et $e^{-iA}$. Ceci permet alors d'avoir à résoudre successivement deux equations plus simples. 
\\Ici, $A=H_0$ et $B=V(x)-\mu(x)\varepsilon(t)$ est la partie potentielle. On obtient l'approximation
\begin{equation}
\psi(x,t+\Delta T) \approx e^{-iH_0 \frac{\Delta T}{2}}e^{-i(V(x)-\mu\varepsilon)\Delta T}e^{-iH_0 \frac{\Delta T}{2}} \psi(x,t)
\end{equation}
\begin{pro}
	Cette méthode conserve la norme $L^2$ de la fonction d’onde $\psi$ au cours du temps.
\end{pro}
\begin{ proof }
	Les opérateurs exponentiés étant antihermitiens, leurs exponentielles sont unitaires. En effet, notons $X^{T}$, $\overline{X}$ et $X^{\dagger}=(\overline{X})^T$ respectivement la transposée, la conjuguée et l'adjointe de $X$. Puisque $e^{X^T}={(e^X)}^T$ et $e^{\overline{X}}=\overline{e^X}$ alors ${e^X}^{\dagger}={(e^X)}^{\dagger}$. Il s'ensuit que si X est antihermitienne, c'est-à-dire $X^{\dagger}=-X$, alors ${(e^X)}^{-1}={(e^X)}^{\dagger}$ : $e^X$ est unitaire.
\end{ proof }
Cette propriété importante \eqref{normecons} est donc conservée après discrétisation.\\
D’autres méthodes de résolution approchée peuvent être envisagées : schémas d’Euler, Runge-Kutta... Le problème posé par ces schémas est qu’ils ne conservent pas la norme de la solution en général.
\subsection{Schéma de Cranck-Nicholson}
Une exception existe avec le schéma implicite de Cranck-Nicholson défini dans notre cas par :
\begin{equation}
i\dfrac{\psi_{j+1}-\psi_j}{\Delta T}=(H_0-\mu\varepsilon_j)\dfrac{\psi_{j+1}+\psi_j}{2}
\end{equation}
En tant que schéma implicite, son inconvénient majeur est d’être coûteux.