\chapter*{Notations}\addcontentsline{toc}{chapter}{Notations}
$
\begin{array}{ll}
\Omega\quad\quad\quad& \mbox{ Domaine de }\mathbb{R}^d,d=1,2.\\\\
T & \mbox{ Maillage }\\\\
(T_i)_{i\in I} & \mbox{ Famille de maillage }\\\\
L^1(\Omega) & \mbox{ L'espace des fonctions intégrables sur }\Omega\\\\
L^2(\Omega) & \mbox{ L'espace des fonctions de carré intégrable sur } \Omega\\\\
H^n(\Omega) & \mbox{ L’espace des fonctions de }L^2(\Omega) \mbox{ dont les dérivées partielles jusqu’à }\\
&\mbox{ l’ordre } n \mbox{ appartiennent à } L^2(\Omega)\\\\
|.| & \mbox{ Valeur absolue}\\\\
|.|_{l,T} & \mbox{ Semi-norme } H^l \mbox{ sur l’ensemble }T\\\\
\|.\| & \mbox{ Norme }\\\\
\|.\|_a & \mbox{ Norme dans l’espace } a\\\\
u_h & \mbox{ Valeur approchée de la fonction } u\\\\
H_u(z) &  \mbox{ Matrice hessienne de la fonction } u \mbox{ au point } z\\\\
\mathcal{H}_u(z) & \mbox{ Valeur approchée de la matrice hessienne de la fonction } u \mbox{ au point } z\\\\
\mathcal{M} & \mbox{ Métrique }\\\\
\left<.,.\right> & \mbox{ Produit scalaire de }\mathbb{R}^d\\\\
\nabla u & \mbox{ Gradient de la fonction } u\\\\
A^T & \mbox{ Transposée de la matrice } A\\\\
\end{array}
$
