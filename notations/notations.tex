\chapter*{Notations}\addcontentsline{toc}{chapter}{Notations}
$
\begin{array}{ll}
\Omega & \mbox{Espace des configurations}\\\\
\mathcal{H} & \mbox{Espace de Hilbert correspondant à un système quantique.}\\
& \mathcal{H} = L^2(\Omega, \C) \mbox{ ou } \C^N  \mbox{ en dimension finie}\\\\
\mathcal{H^*} & \mbox{Dual de } \mathcal{H}\\\\
\lVert\cdot\rVert &  \mbox{norme associée à } \mathcal{H}\\\\
\langle\cdot,\cdot\rangle & \mbox{produit hermitien sur } \mathcal{H}\\
& \langle\psi,\phi\rangle=\sum_{k=1}^{N} \psi^*_k \phi_k \mbox{ en dimension finie}\\
& \langle\psi,\phi\rangle=\int_{\Omega} \psi^*(x) \phi(x) dx \mbox{ en dimension infinie}\\
&\mbox{on note indifféremment } \langle\psi,\phi\rangle = \langle\psi||\phi\rangle = \langle\psi|\phi\rangle\\\\
| \phi \rangle & \mbox{notation ket de Dirac}\\
& \mbox{dans ce memoire on note indifféremment } \phi = | \phi \rangle \in \mathcal{H}\\\\
\langle \psi | & \mbox{notation bra de Dirac}\\
& \mbox{dans ce memoire on note indifféremment } \psi = \langle  \psi | \in \mathcal{H^*}\\\\
L^2(\Omega) & \mbox{ L'espace des fonctions de carré intégrable sur } \Omega\\\\
H^n(\Omega) & \mbox{ L’espace des fonctions de }L^2(\Omega) \mbox{ dont les dérivées partielles jusqu’à }\\
&\mbox{ l’ordre } n \mbox{ appartiennent à } L^2(\Omega)\\\\
|\cdot| & \mbox{ Valeur absolue}\\\\
\left(\cdot,\cdot \right) & \mbox{ Produit scalaire de }\mathbb{R}^d\\\\
\nabla u & \mbox{ Gradient de la fonction } u\\\\
A^T & \mbox{ Transposée de la matrice } A\\\\
\Re & \mbox{Partie réelle}\\\\
\Im & \mbox{Partie imaginaire}
\end{array}
$
